%%%%%%%%%%%%%%%%%%%%%%%%%%%%%%%%%%%%%%%%%
% Beamer Presentation
% LaTeX Template
% Version 1.0 (10/11/12)
%
% This template has been downloaded from:
% http://www.LaTeXTemplates.com
%
% License:
% CC BY-NC-SA 3.0 (http://creativecommons.org/licenses/by-nc-sa/3.0/)
%
%%%%%%%%%%%%%%%%%%%%%%%%%%%%%%%%%%%%%%%%%

%----------------------------------------------------------------------------------------
%	PACKAGES AND THEMES
%----------------------------------------------------------------------------------------

%\documentclass{beamer}
%\usepackage{enumitem}
%\usepackage{booktabs}  
\documentclass{beamer} %
\usetheme{metropolis}
\usepackage[latin1]{inputenc}
\usefonttheme{professionalfonts}
\usepackage{times}
\usepackage{tikz}
\usepackage{amsmath}
\usepackage{verbatim}
\usetikzlibrary{arrows,shapes}   
\def \S {\mathbf{S}}
\def \A {\mathcal{A}}
\def \X {\mathcal{X}}
\def \Ab {\bar{\A}}
\def \R {\mathbb{R}}
\def \Kt {\widetilde{K}}
\def \k {\mathbf{k}}
\def \w {\mathbf{w}}
\def \v {\mathbf{v}}
\def \t {\mathbf{t}}
\def \x {\mathbf{x}}
\def \Se {\mathcal{S}}
\def \E {\mathrm{E}}
\def \Rh {\widehat{R}}
\def \x {\mathbf{x}}
\def \p {\mathbf{p}}
\def \a {\mathbf{a}}
\def \diag {\mbox{diag}}
\def \b {\mathbf{b}}
\def \e {\mathbf{e}}
\def \ba {\boldsymbol{\alpha}}
\def \c {\mathbf{c}}
\def \tr {\mbox{tr}}
\def \d {\mathbf{d}}
\def \z {\mathbf{z}}
\def \s {\mathbf{s}}
\def \bh {\widehat{b}}
\def \y {\mathbf{y}}
\def \u {\mathbf{u}}
%\def \L {\mathcal{L}}
\def \H {\mathcal{H}}
\def \g {\mathbf{g}}
\def \F {\mathcal{F}}
\def \I {\mathbb{I}}
\def \P {\mathcal{P}}
\def \Q {\mathcal{Q}}
\def \xh {\widehat{\x}}
\def \wh {\widehat{\w}}
\def \ah {\widehat{\alpha}}
\def \Rc {\mathcal R}

\def \Bh {\widehat B}
\def \Ah {\widehat A}
\def \Uh {\widehat U}
\def \Ut {\widetilde U}
\def \B {\mathbf B}
\def \C {\mathbf C}
\def \U {\mathbf U}
\def \Kh {\widehat K}
\def \fh {\widehat f}
\def \yh {\widehat y}
\def \Xh {\widehat{X}}
\def \Fh {\widehat{F}}


\def \y {\mathbf{y}}
\def \E {\mathrm{E}}
\def \x {\mathbf{x}}
\def \g {\mathbf{g}}
\def \D {\mathcal{D}}
\def \z {\mathbf{z}}
\def \u {\mathbf{u}}
\def \H {\mathcal{H}}
\def \Pc {\mathcal{P}}
\def \w {\mathbf{w}}
\def \r {\mathbf{r}}
\def \R {\mathbb{R}}
\def \S {\mathcal{S}}
\def \regret {\mbox{regret}}
\def \Uh {\widehat{U}}
\def \Q {\mathcal{Q}}
\def \W {\mathcal{W}}
\def \N {\mathcal{N}}
\def \A {\mathcal{A}}
\def \q {\mathbf{q}}
\def \v {\mathbf{v}}
\def \M {\mathcal{M}}
\def \c {\mathbf{c}}
\def \ph {\widehat{p}}
\def \d {\mathbf{d}}
\def \p {\mathbf{p}}
\def \q {\mathbf{q}}
\def \db {\bar{\d}}
\def \dbb {\bar{d}}
\def \I {\mathcal{I}}
\def \xt {\widetilde{\x}}
\def \f {\mathbf{f}}
\def \a {\mathbf{a}}
\def \b {\mathbf{b}}
\def \ft {\widetilde{\f}}
\def \bt {\widetilde{\b}}
\def \h {\mathbf{h}}
\def \B {\mathbf{B}}
\def \bts {\widetilde{b}}
\def \fts {\widetilde{f}}
\def \Gh {\widehat{G}}
\def \G {\mathcal {G}}
\def \bh {\widehat{b}}
\def \fh {\widehat{f}}
\def \wh {\widehat{\w}}
\def \vb {\bar{v}}
\def \zt {\widetilde{\z}}
\def \zts {\widetilde{z}}
\def \s {\mathbf{s}}
\def \gh {\widehat{\g}}
\def \vh {\widehat{\v}}
\def \Sh {\widehat{S}}
\def \rhoh {\widehat{\rho}}
\def \hh {\widehat{\h}}
\def \C {\mathcal{C}}
\def \V {\mathcal{L}}
\def \t {\mathbf{t}}
\def \xh {\widehat{\x}}
\def \Ut {\widetilde{U}}
\def \wt {\widetilde{\w}}
\def \Th {\widehat{T}}
\def \Ot {\tilde{\mathcal{O}}}
\def \X {\mathcal{X}}
\def \nb {\widehat{\nabla}}
\def \K {\mathcal{K}}
\def \P {\mathbb{P}}
\def \T {\mathcal{T}}
\def \F {\mathcal{F}}
\def \ft{\widetilde{f}}
\def \xt {\widetilde{x}}
\def \Rt {\mathcal{R}}
\def \Rb {\bar{\Rt}}
\def \wb {\bar{\w}}
%\mode<presentation> {
	
	% The Beamer class comes with a number of default slide themes
	% which change the colors and layouts of slides. Below this is a list
	% of all the themes, uncomment each in turn to see what they look like.
	
	%\usetheme{default}
	%\usetheme{AnnArbor}
	%\usetheme{Antibes}
	%\usetheme{Bergen}
	%\usetheme{Berkeley}
	%\usetheme{Berlin}
	%\usetheme{Boadilla}
	%\usetheme{CambridgeUS}
	%\usetheme{Copenhagen}
	%\usetheme{Darmstadt}
	%\usetheme{Dresden}
	%\usetheme{Frankfurt}
	%\usetheme{Goettingen}
	%\usetheme{Hannover}
	%\usetheme{Ilmenau}
	%\usetheme{JuanLesPins}
	%\usetheme{Luebeck}
	%\usetheme{Madrid}
	%\usetheme{Malmoe}
	%\usetheme{Marburg}
	%\usetheme{Montpellier}
	%\usetheme{PaloAlto}
	%\usetheme{Pittsburgh}
	%\usetheme{Rochester}
	%\usetheme{Singapore}
	%\usetheme{Szeged}
	%\usetheme{Warsaw}
	
	% As well as themes, the Beamer class has a number of color themes
	% for any slide theme. Uncomment each of these in turn to see how it
	% changes the colors of your current slide theme.
	
	%\usecolortheme{albatross}
	%\usecolortheme{beaver}
	%\usecolortheme{beetle}
	%\usecolortheme{crane}
	%\usecolortheme{dolphin}
	%\usecolortheme{dove}
	%\usecolortheme{fly}
	%\usecolortheme{lily}
	%\usecolortheme{orchid}
	%\usecolortheme{rose}
	%\usecolortheme{seagull}
	%\usecolortheme{seahorse}
	%\usecolortheme{whale}
	%\usecolortheme{wolverine}
	
	%\setbeamertemplate{footline} % To remove the footer line in all slides uncomment this line
	%\setbeamertemplate{footline}[page number] % To replace the footer line in all slides with a simple slide count uncomment this line
	
	%\setbeamertemplate{navigation symbols}{} % To remove the navigation symbols from the bottom of all slides uncomment this line
%}

\usepackage{graphicx} % Allows including images
\usepackage{booktabs} % Allows the use of \toprule, \midrule and \bottomrule in tables

%----------------------------------------------------------------------------------------
%	TITLE PAGE
%----------------------------------------------------------------------------------------

\title{Efficient Distributed Stochastic Dual Coordinate Ascent} % The short title appears at the bottom of every slide, the full title is only on the title page

%\author{John Smith} % Your name
\author{Jeff Hajewski \\ Mingrui Liu}
\date{May 3, 2017}
\institute{University of Iowa}

%\date{\today} % Date, can be changed to a custom date
\AtBeginSubsection[]
{
	\begin{frame}<beamer>{Outline}
		\tableofcontents[currentsection,currentsubsection]
	\end{frame}
}
\begin{document}
	
	\begin{frame}
		\titlepage % Print the title page as the first slide
	\end{frame}
	
	\begin{frame}
		\frametitle{Overview} % Table of contents slide, comment this block out to remove it
		\tableofcontents % Throughout your presentation, if you choose to use \section{} and \subsection{} commands, these will automatically be printed on this slide as an overview of your presentation
	\end{frame}
	
	%----------------------------------------------------------------------------------------
	%	PRESENTATION SLIDES
	%----------------------------------------------------------------------------------------
	
	%------------------------------------------------
\section{Problem Description and Related Work}
\subsection{Problem of Interest}
\begin{frame}{Problem of Interest}
	Many machine learning problems can be formulated as the Regularized Finite Sum Minimization (RFSM) problem.
	\begin{equation}
		\label{RLM}
		\min_{w\in\R^d}P(w), \text{where }P(w)=\frac{1}{n}\sum_{i=1}^{n}\phi(w^\top x_i,y_i)+\lambda g(w),
	\end{equation}
	where $w\in\R^d$ denotes the weight vector, $(x_i,y_i),x_i\in\R^d,y_i\in\R$,
	$i=1,\ldots,n$ are training data, $\lambda>0$ is a regularization parameter,
	$\phi(z,y)$ is a convex function of $z$, and $g(w)$ is a convex function of $w$.
\end{frame}

\begin{frame}{Approaches to solve RFSM problem}
	\begin{itemize}
	\item \textbf{The difficulty:}\\ When the data size $n$ is very large, it is difficult to use full gradient method or even fit all data on one single machine.
	\pause
	
	\item \textbf{The countermeasures:}\\
	\begin{itemize}
		\item Stochastic Optimization\\
		\item Distributed Optimization
	\end{itemize}
	\end{itemize}
\end{frame}
%
\subsection{Related Work}
\begin{frame}{Stochastic Optimization}
	\begin{itemize}
	\item	Stochastic Gradient Descent (SGD) \cite{bottou2010large,nemirovski2009robust}
	%\pause
	\item Stochastic Variance Reduced Gradient (SVRG) \cite{johnson2013accelerating,xiao2014proximal}
	%\pause
	\item  \textcolor{red}{Stochastic Dual Coordinate Ascent (SDCA)} \cite{shalev2013stochastic,shalev2014accelerated}
	%\pause
	\item \ldots
	\end{itemize}
\end{frame}
	%----------------------------------------------------------------------------------------
\begin{frame}{Distributed Optimization}
	\begin{itemize}
		\item Distributed SGD \cite{lian2015asynchronous}
		\item Distributed Stochastic ADMM \cite{boyd2011distributed}
		\item \textcolor{red}{Distributed SDCA}  \cite{yang2013trading,yang2013analysis}
	\end{itemize}
\end{frame}
\begin{frame}{Our Contribution}
	\begin{itemize}
	\item The current SDCA and distributed SDCA are implemented by CPU
	\item Our contribution is to implement \textcolor{red}{a practically more efficient GPU-based implementation}, in both sequential setting and distributed setting.
	\end{itemize}
\end{frame}
\section{Practical GPU-version of SDCA}
\subsection{GPU-version of vanilla SDCA}
\begin{frame}{GPU-version vanilla SDCA}

\end{frame}
\bibliographystyle{apalike}	
\bibliography{courseProject}

\end{document}


%\AtBeginSection[]
%{
%	\begin{frame}<beamer>
%		\frametitle{Outline for section \thesection}
%		\tableofcontents[currentsection]
%	\end{frame}
%}
%
%\title{Efficient Distributed SDCA}
%
%\begin{document}
%\maketitle
%\begin{frame}{Table of contents}
%\tableofcontents
%\end{frame}
%\section{Overview}


%\end{document}
