%%%%%%%%%%%%%%%%%%%%%%%%%%%%%%%%%%%%%%%%%
% Beamer Presentation
% LaTeX Template
% Version 1.0 (10/11/12)
%
% This template has been downloaded from:
% http://www.LaTeXTemplates.com
%
% License:
% CC BY-NC-SA 3.0 (http://creativecommons.org/licenses/by-nc-sa/3.0/)
%
%%%%%%%%%%%%%%%%%%%%%%%%%%%%%%%%%%%%%%%%%

%----------------------------------------------------------------------------------------
%	PACKAGES AND THEMES
%----------------------------------------------------------------------------------------

%\documentclass{beamer}
%\usepackage{enumitem}
%\usepackage{booktabs}  
\documentclass{beamer} 
\usetheme{metropolis}
\usepackage{algorithmicx}
\usepackage[latin1]{inputenc}
\usefonttheme{professionalfonts}
\usepackage{times}
\usepackage{tikz}
\usepackage{amsmath}
\usepackage{fancyvrb}
\usepackage{listings}
\usetikzlibrary{arrows,shapes}   
\def \S {\mathbf{S}}
\def \A {\mathcal{A}}
\def \X {\mathcal{X}}
\def \Ab {\bar{\A}}
\def \R {\mathbb{R}}
\def \Kt {\widetilde{K}}
\def \k {\mathbf{k}}
\def \w {\mathbf{w}}
\def \v {\mathbf{v}}
\def \t {\mathbf{t}}
\def \x {\mathbf{x}}
\def \Se {\mathcal{S}}
\def \E {\mathrm{E}}
\def \Rh {\widehat{R}}
\def \x {\mathbf{x}}
\def \p {\mathbf{p}}
\def \a {\mathbf{a}}
\def \diag {\mbox{diag}}
\def \b {\mathbf{b}}
\def \e {\mathbf{e}}
\def \ba {\boldsymbol{\alpha}}
\def \c {\mathbf{c}}
\def \tr {\mbox{tr}}
\def \d {\mathbf{d}}
\def \z {\mathbf{z}}
\def \s {\mathbf{s}}
\def \bh {\widehat{b}}
\def \y {\mathbf{y}}
\def \u {\mathbf{u}}
%\def \L {\mathcal{L}}
\def \H {\mathcal{H}}
\def \g {\mathbf{g}}
\def \F {\mathcal{F}}
\def \I {\mathbb{I}}
\def \P {\mathcal{P}}
\def \Q {\mathcal{Q}}
\def \xh {\widehat{\x}}
\def \wh {\widehat{\w}}
\def \ah {\widehat{\alpha}}
\def \Rc {\mathcal R}

\def \Bh {\widehat B}
\def \Ah {\widehat A}
\def \Uh {\widehat U}
\def \Ut {\widetilde U}
\def \B {\mathbf B}
\def \C {\mathbf C}
\def \U {\mathbf U}
\def \Kh {\widehat K}
\def \fh {\widehat f}
\def \yh {\widehat y}
\def \Xh {\widehat{X}}
\def \Fh {\widehat{F}}


\def \y {\mathbf{y}}
\def \E {\mathrm{E}}
\def \x {\mathbf{x}}
\def \g {\mathbf{g}}
\def \D {\mathcal{D}}
\def \z {\mathbf{z}}
\def \u {\mathbf{u}}
\def \H {\mathcal{H}}
\def \Pc {\mathcal{P}}
\def \w {\mathbf{w}}
\def \r {\mathbf{r}}
\def \R {\mathbb{R}}
\def \S {\mathcal{S}}
\def \regret {\mbox{regret}}
\def \Uh {\widehat{U}}
\def \Q {\mathcal{Q}}
\def \W {\mathcal{W}}
\def \N {\mathcal{N}}
\def \A {\mathcal{A}}
\def \q {\mathbf{q}}
\def \v {\mathbf{v}}
\def \M {\mathcal{M}}
\def \c {\mathbf{c}}
\def \ph {\widehat{p}}
\def \d {\mathbf{d}}
\def \p {\mathbf{p}}
\def \q {\mathbf{q}}
\def \db {\bar{\d}}
\def \dbb {\bar{d}}
\def \I {\mathcal{I}}
\def \xt {\widetilde{\x}}
\def \f {\mathbf{f}}
\def \a {\mathbf{a}}
\def \b {\mathbf{b}}
\def \ft {\widetilde{\f}}
\def \bt {\widetilde{\b}}
\def \h {\mathbf{h}}
\def \B {\mathbf{B}}
\def \bts {\widetilde{b}}
\def \fts {\widetilde{f}}
\def \Gh {\widehat{G}}
\def \G {\mathcal {G}}
\def \bh {\widehat{b}}
\def \fh {\widehat{f}}
\def \wh {\widehat{\w}}
\def \vb {\bar{v}}
\def \zt {\widetilde{\z}}
\def \zts {\widetilde{z}}
\def \s {\mathbf{s}}
\def \gh {\widehat{\g}}
\def \vh {\widehat{\v}}
\def \Sh {\widehat{S}}
\def \rhoh {\widehat{\rho}}
\def \hh {\widehat{\h}}
\def \C {\mathcal{C}}
\def \V {\mathcal{L}}
\def \t {\mathbf{t}}
\def \xh {\widehat{\x}}
\def \Ut {\widetilde{U}}
\def \wt {\widetilde{\w}}
\def \Th {\widehat{T}}
\def \Ot {\tilde{\mathcal{O}}}
\def \X {\mathcal{X}}
\def \nb {\widehat{\nabla}}
\def \K {\mathcal{K}}
\def \P {\mathbb{P}}
\def \T {\mathcal{T}}
\def \F {\mathcal{F}}
\def \ft{\widetilde{f}}
\def \xt {\widetilde{x}}
\def \Rt {\mathcal{R}}
\def \Rb {\bar{\Rt}}
\def \wb {\bar{\w}}
%\mode<presentation> {
	
	% The Beamer class comes with a number of default slide themes
	% which change the colors and layouts of slides. Below this is a list
	% of all the themes, uncomment each in turn to see what they look like.
	
	%\usetheme{default}
	%\usetheme{AnnArbor}
	%\usetheme{Antibes}
	%\usetheme{Bergen}
	%\usetheme{Berkeley}
	%\usetheme{Berlin}
	%\usetheme{Boadilla}
	%\usetheme{CambridgeUS}
	%\usetheme{Copenhagen}
	%\usetheme{Darmstadt}
	%\usetheme{Dresden}
	%\usetheme{Frankfurt}
	%\usetheme{Goettingen}
	%\usetheme{Hannover}
	%\usetheme{Ilmenau}
	%\usetheme{JuanLesPins}
	%\usetheme{Luebeck}
	%\usetheme{Madrid}
	%\usetheme{Malmoe}
	%\usetheme{Marburg}
	%\usetheme{Montpellier}
	%\usetheme{PaloAlto}
	%\usetheme{Pittsburgh}
	%\usetheme{Rochester}
	%\usetheme{Singapore}
	%\usetheme{Szeged}
	%\usetheme{Warsaw}
	
	% As well as themes, the Beamer class has a number of color themes
	% for any slide theme. Uncomment each of these in turn to see how it
	% changes the colors of your current slide theme.
	
	%\usecolortheme{albatross}
	%\usecolortheme{beaver}
	%\usecolortheme{beetle}
	%\usecolortheme{crane}
	%\usecolortheme{dolphin}
	%\usecolortheme{dove}
	%\usecolortheme{fly}
	%\usecolortheme{lily}
	%\usecolortheme{orchid}
	%\usecolortheme{rose}
	%\usecolortheme{seagull}
	%\usecolortheme{seahorse}
	%\usecolortheme{whale}
	%\usecolortheme{wolverine}
	
	%\setbeamertemplate{footline} % To remove the footer line in all slides uncomment this line
	%\setbeamertemplate{footline}[page number] % To replace the footer line in all slides with a simple slide count uncomment this line
	
	%\setbeamertemplate{navigation symbols}{} % To remove the navigation symbols from the bottom of all slides uncomment this line
%}

\usepackage{graphicx} % Allows including images
\usepackage{booktabs} % Allows the use of \toprule, \midrule and \bottomrule in tables

%----------------------------------------------------------------------------------------
%	TITLE PAGE
%----------------------------------------------------------------------------------------

\title{Efficient Distributed Stochastic Dual Coordinate Ascent} % The short title appears at the bottom of every slide, the full title is only on the title page

%\author{John Smith} % Your name
\author{Jeff Hajewski \\ Mingrui Liu}
\date{May 3, 2017}
\institute{University of Iowa}

%\date{\today} % Date, can be changed to a custom date
\AtBeginSubsection[]
{
	\begin{frame}<beamer>{Outline}
		\tableofcontents[currentsection,currentsubsection]
	\end{frame}
}
\begin{document}
	
	\begin{frame}
		\titlepage % Print the title page as the first slide
	\end{frame}
	
	%\begin{frame}
	%	\frametitle{Overview} % Table of contents slide, comment this block out to remove it
	%	\tableofcontents % Throughout your presentation, if you choose to use \section{} and \subsection{} commands, these will automatically be printed on this slide as an overview of your presentation
	%\end{frame}
	
	%----------------------------------------------------------------------------------------
	%	PRESENTATION SLIDES
	%----------------------------------------------------------------------------------------
	
	%------------------------------------------------
\section{Problem Overview}
\begin{frame}{Problem Overview - The Primal Problem}
	Many machine learning problems can be formulated as the Regularized Finite Sum Minimization (RFSM) problem.
  \vspace{1em}
	\begin{equation}
		\label{RLM}
		\min_{w\in\R^d}P(w)
	\end{equation}
  \vspace{1.5em}\\
  where 
  \begin{align*}
    P(w) &= \frac{1}{n}\sum_{i=1}^{n}\phi(w^\top x_i,y_i)+\lambda g(w)\\
    w, x_i & \in \R^d, \text{ for } i = 1, \dots, n\\
    y_i & \in \R, \text{ for } i = 1, \dots, n\\
    \phi(z,y) & \text{ is convex in $z$}\\
    g(w) & \text{ is convex in $w$}\\
  \end{align*}
  
\end{frame}

\begin{frame}{Problem Overview - The Dual Problem}
  We consider the case when $g(w)=\frac{1}{2}\|w\|_2^2$, then the dual problem is given by
  \vspace{1em}
  \begin{equation}
    \max_{\alpha \in \R^n} D(\alpha)
  \end{equation}
  where 
  \begin{align*}
    D(\alpha) &= \frac{1}{n}\sum_{i=1}^n -\phi^*_i(-\alpha_i) -
    \frac{\lambda}{2}\Big|\Big|\frac{1}{\lambda n}\sum_{i=1}^n\alpha_i x_i\Big|\Big|^2\\
    x_i & \in \R^d, \text{ for } i = 1, \dots, n\\
    \alpha & \in \R^n\\
    \phi^*_i(u) &= \max_z (zu - \phi_i(z))
  \end{align*}
\end{frame}
%
%\subsection{Related Work}
\begin{frame}{Related Work}
  The two key papers influencing our work are:
	\begin{itemize}
	  \item  Stochastic Dual Coordinate Ascent (SDCA) \cite{shalev2013stochastic}
	  \item Distributed SDCA  \cite{yang2013trading,yang2013analysis}
	\end{itemize}
\end{frame}
	%----------------------------------------------------------------------------------------

\begin{frame}{What is SDCA?}
  \begin{itemize}
    \item \textbf{SDCA} - randomly pick a coordinate axis of $\mathbf{\alpha}
      \in \R^n$, find update that best improves
      the objective\vspace{1em}
    \item \textbf{Distributed SDCA} - randomly pick $k$ coordinate axes of
      $\mathbf{\alpha} \in \R^n$, simultaneously
      find updates that best improve the objective (independently)
  \end{itemize}
\end{frame}

\begin{frame}[standout]
  What if we ran SDCA on the GPU?
\end{frame}

\begin{frame}{Parallelizing SDCA}
  Two approaches:
	\begin{itemize}
    \item Naive approach: simply parallelize all tensor operations (e.g., dot
      product, matrix-vector multiplication, etc.) \pause
  \item Better: mimic the distributed apporach by \cite{yang2013trading} on a GPU
	\end{itemize}
\end{frame}
\section{Implementation}
\begin{frame}{Main Points}
  The key area of concern:
  \begin{itemize}
    \item Memory
      \begin{itemize}
        \item Allocation \pause
        \item Communication (via PCIE bus rather than network) \pause
        \item How can we cognitive load of writing this code?
      \end{itemize}
  \end{itemize}
\end{frame}

\begin{frame}{Dealing with Memory Allocation}
  Naive approach:
  \VerbatimInput[fontsize=\small]{allocExample.txt}
  On a small dataset(200 points in $\mathbb{R}^3$) we hit over 200k allocations,
  which comprised nearly \textbf{95\%} of the GPU compute time ($\approx 13$
    seconds).
\end{frame}

\begin{frame}[standout]
  Can we do better?\\ \pause
  Yes!
\end{frame}


\begin{frame}{Dealing with Memory Allocation}
  Better approach:
  \VerbatimInput[fontsize=\small]{allocImproved.txt}
  The use of static class pointers reduced the 200k allocations down to only
  \textbf{3 memory allocations}, which comprised only \textbf{0.03\%} compute
    time ($\approx 180 \mu s$).
\end{frame}

\begin{frame}[standout]
  What about the cost of communication?
\end{frame}

\begin{frame}{Communication Costs}
  Copying data to and from the GPU is the next most expensive operation.
  \begin{itemize}
    \item About 50\% of the compute time, or 768 ms (using the same toy dataset,
      after we have fixed the memory allocation issue) \pause
    \item Mostly unnecessary!
  \end{itemize}
\end{frame}

\begin{frame}{Communication Costs}
  Consider the following algorithm using the GPU:
  \begin{algorithmic}[1]
    \State $\Delta \omega_i \gets f(\mathbf{x}, \omega)$
    \State $\omega_i \gets \omega_i + \Delta \omega_i$
  \end{algorithmic}
  
  To handle this efficiently we should:
  \begin{itemize}
      \pause
    \item Reuse $\omega$ in step 2 since we already moved it to the GPU for step
      1 \pause
    \item Perform step 2 on the GPU since the data is already there. No need to
      pull it off and then move it back to the GPU
  \end{itemize}
\end{frame}

\begin{frame}{Communication Costs}
  The sad reality is this is quite complicated.

  \begin{itemize}
    \item Lots of book-keeping
    \item Are there edge cases?
    \item Need to watch out for memory leaks. Remember, \textbf{no} GC!
  \end{itemize}
\end{frame}

\begin{frame}[standout]
  How can we handle this complexity?
\end{frame}

\begin{frame}{Wrappers}
  We use wrappers (also known as decorators) to add additional functionality to
  our code. For example,
  \VerbatimInput[fontsize=\small]{wrapperEx.txt}
\end{frame}

\begin{frame}{Wrappers}
  To handle the flow of data from GPU to CPU, as well as book-keeping, we can
  use something like this:
  \VerbatimInput[fontsize=\small]{dataEx.txt}
\end{frame}

\begin{frame}[standout]
  What about results?
\end{frame}

\begin{frame}
  We are still finalizing results. There is a lot of non-trivial structural code
  behind this.
  \begin{itemize}
    \item Loading data \pause
    \item CUDA can be challenging \pause
    \item Algorithm \pause
    \item Algorithm performance tracking \pause
    \item Expect distributed SDCA to be the fastest, followed by CUDA
      accelerated SDCA, followed by SDCA
  \end{itemize}
\end{frame}

\bibliographystyle{apalike}	
\bibliography{courseProject}

\end{document}


%\AtBeginSection[]
%{
%	\begin{frame}<beamer>
%		\frametitle{Outline for section \thesection}
%		\tableofcontents[currentsection]
%	\end{frame}
%}
%
%\title{Efficient Distributed SDCA}
%
%\begin{document}
%\maketitle
%\begin{frame}{Table of contents}
%\tableofcontents
%\end{frame}
%\section{Overview}


%\end{document}
