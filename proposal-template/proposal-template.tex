\documentclass{article}

% if you need to pass options to natbib, use, e.g.:
% \PassOptionsToPackage{numbers, compress}{natbib}
% before loading nips_2016
%
% to avoid loading the natbib package, add option nonatbib:
% \usepackage[nonatbib]{nips_2016}

%\usepackage{nips_2016}

% to compile a camera-ready version, add the [final] option, e.g.:
 \usepackage[final]{nips_2016}

\usepackage[utf8]{inputenc} % allow utf-8 input
\usepackage[T1]{fontenc}    % use 8-bit T1 fonts
\usepackage{hyperref}       % hyperlinks
\usepackage{url}            % simple URL typesetting
\usepackage{booktabs}       % professional-quality tables
\usepackage{amsfonts}       % blackboard math symbols
\usepackage{nicefrac}       % compact symbols for 1/2, etc.
\usepackage{microtype}      % microtypography
\usepackage{amsmath}
\usepackage{algorithm}
\usepackage{algorithmic}
\usepackage{times}
% For figures
\usepackage{graphicx} % more modern
%\usepackage{epsfig} % less modern
\usepackage{subfigure} 
%\usepackage{subcaption}
% For citations
\usepackage{natbib}

\usepackage{amsmath, amsthm, amssymb, multirow, paralist}
\newtheorem{thm}{Theorem}
\newtheorem{prop}{Proposition}
\newtheorem{lemma}{Lemma}
\newtheorem{cor}{Corollary}
\newtheorem{definition}{Definition}
%\newtheorem{ass}[thm]{Assumption}
\newtheorem{assumption}{Assumption}
\newtheorem{obs}{Observation}

\usepackage{enumitem}
\usepackage{booktabs}     
\def \S {\mathbf{S}}
\def \A {\mathcal{A}}
\def \X {\mathcal{X}}
\def \Ab {\bar{\A}}
\def \R {\mathbb{R}}
\def \Kt {\widetilde{K}}
\def \k {\mathbf{k}}
\def \w {\mathbf{w}}
\def \v {\mathbf{v}}
\def \t {\mathbf{t}}
\def \x {\mathbf{x}}
\def \Se {\mathcal{S}}
\def \E {\mathrm{E}}
\def \Rh {\widehat{R}}
\def \x {\mathbf{x}}
\def \p {\mathbf{p}}
\def \a {\mathbf{a}}
\def \diag {\mbox{diag}}
\def \b {\mathbf{b}}
\def \e {\mathbf{e}}
\def \ba {\boldsymbol{\alpha}}
\def \c {\mathbf{c}}
\def \tr {\mbox{tr}}
\def \d {\mathbf{d}}
\def \z {\mathbf{z}}
\def \s {\mathbf{s}}
\def \bh {\widehat{b}}
\def \y {\mathbf{y}}
\def \u {\mathbf{u}}
%\def \L {\mathcal{L}}
\def \H {\mathcal{H}}
\def \g {\mathbf{g}}
\def \F {\mathcal{F}}
\def \I {\mathbb{I}}
\def \P {\mathcal{P}}
\def \Q {\mathcal{Q}}
\def \xh {\widehat{\x}}
\def \wh {\widehat{\w}}
\def \ah {\widehat{\alpha}}
\def \Rc {\mathcal R}

\def \Bh {\widehat B}
\def \Ah {\widehat A}
\def \Uh {\widehat U}
\def \Ut {\widetilde U}
\def \B {\mathbf B}
\def \C {\mathbf C}
\def \U {\mathbf U}
\def \Kh {\widehat K}
\def \fh {\widehat f}
\def \yh {\widehat y}
\def \Xh {\widehat{X}}
\def \Fh {\widehat{F}}


\def \y {\mathbf{y}}
\def \E {\mathrm{E}}
\def \x {\mathbf{x}}
\def \g {\mathbf{g}}
\def \D {\mathcal{D}}
\def \z {\mathbf{z}}
\def \u {\mathbf{u}}
\def \H {\mathcal{H}}
\def \Pc {\mathcal{P}}
\def \w {\mathbf{w}}
\def \r {\mathbf{r}}
\def \R {\mathbb{R}}
\def \S {\mathcal{S}}
\def \regret {\mbox{regret}}
\def \Uh {\widehat{U}}
\def \Q {\mathcal{Q}}
\def \W {\mathcal{W}}
\def \N {\mathcal{N}}
\def \A {\mathcal{A}}
\def \q {\mathbf{q}}
\def \v {\mathbf{v}}
\def \M {\mathcal{M}}
\def \c {\mathbf{c}}
\def \ph {\widehat{p}}
\def \d {\mathbf{d}}
\def \p {\mathbf{p}}
\def \q {\mathbf{q}}
\def \db {\bar{\d}}
\def \dbb {\bar{d}}
\def \I {\mathcal{I}}
\def \xt {\widetilde{\x}}
\def \f {\mathbf{f}}
\def \a {\mathbf{a}}
\def \b {\mathbf{b}}
\def \ft {\widetilde{\f}}
\def \bt {\widetilde{\b}}
\def \h {\mathbf{h}}
\def \B {\mathbf{B}}
\def \bts {\widetilde{b}}
\def \fts {\widetilde{f}}
\def \Gh {\widehat{G}}
\def \G {\mathcal {G}}
\def \bh {\widehat{b}}
\def \fh {\widehat{f}}
\def \wh {\widehat{\w}}
\def \vb {\bar{v}}
\def \zt {\widetilde{\z}}
\def \zts {\widetilde{z}}
\def \s {\mathbf{s}}
\def \gh {\widehat{\g}}
\def \vh {\widehat{\v}}
\def \Sh {\widehat{S}}
\def \rhoh {\widehat{\rho}}
\def \hh {\widehat{\h}}
\def \C {\mathcal{C}}
\def \V {\mathcal{L}}
\def \t {\mathbf{t}}
\def \xh {\widehat{\x}}
\def \Ut {\widetilde{U}}
\def \wt {\widetilde{\w}}
\def \Th {\widehat{T}}
\def \Ot {\tilde{\mathcal{O}}}
\def \X {\mathcal{X}}
\def \nb {\widehat{\nabla}}
\def \K {\mathcal{K}}
\def \P {\mathbb{P}}
\def \T {\mathcal{T}}
\def \F {\mathcal{F}}
\def \ft{\widetilde{f}}
\def \xt {\widetilde{x}}
\def \Rt {\mathcal{R}}
\def \Rb {\bar{\Rt}}
\def \wb {\bar{\w}}
\title{Efficient Distributed Stochastic Dual Coordinate Ascent}

% The \author macro works with any number of authors. There are two
% commands used to separate the names and addresses of multiple
% authors: \And and \AND.
%
% Using \And between authors leaves it to LaTeX to determine where to
% break the lines. Using \AND forces a line break at that point. So,
% if LaTeX puts 3 of 4 authors names on the first line, and the last
% on the second line, try using \AND instead of \And before the third
% author name.

\author{
  Mingrui Liu, Jeff Hajewski \\
  Department of Computer Science\\
  University of Iowa\\
  Iowa City, IA  52242 \\
  \texttt{mingrui-liu@uiowa.edu, jeffery-hajewski@uiowa.edu} \\
  %% examples of more authors
%   \And
%   123\\Department of Computer Science\\
%   	University of Iowa\\
%   	Iowa City, IA  52242 \\
%   	\texttt{mingrui-liu@uiowa.edu, jeffery-hajewski@uiowa.edu} \\
  %%Affiliation{Department of Computer Science} \\
  %% Address \\
  %% \texttt{email} \\
  %% \AND
  %% Coauthor \\
  %% Affiliation \\
  %% Address \\
  %% \texttt{email} \\
  %% \And
  %% Coauthor \\
  %% Affiliation \\
  %% Address \\
  %% \texttt{email} \\
  %% \And
  %% Coauthor \\
  %% Affiliation \\
  %% Address \\
  %% \texttt{email} \\
}

\begin{document}
% \nipsfinalcopy is no longer used

\maketitle

\begin{abstract}
  The abstract paragraph should be indented \nicefrac{1}{2}~inch
  (3~picas) on both the left- and right-hand margins. Use 10~point
  type, with a vertical spacing (leading) of 11~points.  The word
  \textbf{Abstract} must be centered, bold, and in point size 12. Two
  line spaces precede the abstract. The abstract must be limited to
  one paragraph. This latex file is modified from the NIPS  2016 template. 
\end{abstract}

\section{Introduction}
%What is the research problem that you are trying to tackle? Describe the background and motivation. 
In recent years, we come into the big data era. Many large-scale machine learning problems, most of which are essentially optimization problems with huge magnitude of data size, need to be tackled. Two common countermeasures to deal with this are employing stochastic optimization algorithms, and utilizing computational resources in a parallel or distributed manner\cite{Boyd2011}. 

In this paper, we consider a class of convex optimization problems with special structure, whose objective can be expressed as the sum of a finite sum of loss functions and a regularization function:
\begin{equation}
\label{RLM}
	\min_{w\in\R^d}F(w), \text{where }P(w)=\frac{1}{n}\sum_{i=1}^{n}\phi(w^\top x_i,y_i)+\lambda g(w),
\end{equation}
where $w\in\R^d$ denotes the weight vector, $(x_i,y_i),x_i\in\R^d,y_i\in\R$, $i=1,\ldots,n$ are training data, $\lambda>0$ is a regularization parameter, $\phi(z,y)$ is a convex function of $z$, and $g(w)$ is a convex function of $w$. We refer to the problem in \ref{RLM} as Regularized Finite Sum Minimization (RFSM) problem. When $g(w)=0$, the problem reduces to the Finite Sum Minimization (FSM) problem.

Both RFSM and FSM problem have been extensively studied in machine learning and optimization literature. When $n$ is large, various stochastic optimization algorithms were proposed\cite{Bottou2010,nemirovski2009robust}.  
\section{Related Work}
\label{gen_inst}
Are there any related work? If yes, review them and discuss their deficiencies and how your proposed work can potentially address their issues.  


\section{The Proposed Work}
Describe your proposed work in this section. 



\section{Plan}
Describe your plan for the project. What data you are going to use to evaluate your methods? What are the baselines that you want to compare? How will you develop your methods?   A timeline with important milestones is always perferred. 


\subsubsection*{Acknowledgments}

Use unnumbered third level headings for the acknowledgments. All
acknowledgments go at the end of the paper. Do not include
acknowledgments in the anonymized submission, only in the final paper.

%\section*{References}

%References follow the acknowledgments. Use unnumbered first-level
%heading for the references. Any choice of citation style is acceptable
%as long as you are consistent. It is permissible to reduce the font
%size to \verb+small+ (9 point) when listing the references.
%
%\medskip
%
%\small
%
%[1] Alexander, J.A.\ \& Mozer, M.C.\ (1995) Template-based algorithms
%for connectionist rule extraction. In G.\ Tesauro, D.S.\ Touretzky and
%T.K.\ Leen (eds.), {\it Advances in Neural Information Processing
%  Systems 7}, pp.\ 609--616. Cambridge, MA: MIT Press.
%
%[2] Bower, J.M.\ \& Beeman, D.\ (1995) {\it The Book of GENESIS:
%  Exploring Realistic Neural Models with the GEneral NEural SImulation
%  System.}  New York: TELOS/Springer--Verlag.
%
%[3] Hasselmo, M.E., Schnell, E.\ \& Barkai, E.\ (1995) Dynamics of
%learning and recall at excitatory recurrent synapses and cholinergic
%modulation in rat hippocampal region CA3. {\it Journal of
%  Neuroscience} {\bf 15}(7):5249-5262.
%123~\cite{Bolte:2006:LIN:1328019.1328299},\cite{banerjee2007generalized}\cite{zhangcs14}\cite{shalev2016sdca}\cite{yang2013trading}
%123~\cite{Shalev-Shwartz2016a} is a good approach
\bibliography{courseProject}
\bibliographystyle{abbrv}
\end{document}